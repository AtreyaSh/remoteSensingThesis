\begin{titlepage}
	
        \vspace*{\fill}
        \vspace{-3cm}
        \begin{minipage}{6.9cm}
        	\hspace{-0.6cm}
        	\includegraphics[width=7cm]{btu.jpg}
        \end{minipage}
       	\hspace{-1.5cm}
        \begin{minipage}{12.0cm}
        	\fontsize{12pt}{1pt}
        	\center{
        		\textbf{Faculty 2:\\         	Environmental and Natural Sciences}\\
        		Chair of Environmental Informatics\\
        	}
        \end{minipage}\\
        \vspace{0.2cm}\\
        \rule{\textwidth}{0.5pt}\\
        \begin{center}
		\LARGE
		\textbf{Remote-Sensing Approach to Spatio-Temporal Analysis of Vegetation Cover in Dharamshala Tehsil Region, Indian Northwestern Himalayas}
		
		\vspace{1cm}
		
		\textbf{Raum-Zeit-Analyse der Vegetationsbedeckung auf der Basis von Fernerkundungsdaten in der Dharamshala Tehsil Region im indischen nordwestlichen Himalaya}
		
		\vspace{1cm}
		
		\LARGE
        Bachelor Thesis Report\\[0.2cm]
        Module 41317
		
		\vspace{1cm}
		
		Atreya Shankar (3433230)\\
		Environmental and Resource Management (ERM) B.Sc.

		\vspace{1cm}
		
	    \Large
		Supervisor: apl. Prof. Dr. -Ing. habil. Frank Molkenthin \\
		Date of Submission: 01 March 2018\\
		\vspace{0.5cm}
		\rule{\textwidth}{0.5pt}\\
		\vspace*{\fill}
	\end{center}
\end{titlepage}

\clearpage

\section*{Abstract}
\pagenumbering{roman} 
\addcontentsline{toc}{section}{Abstract}

\justify
Within the last centuries, the world has seen a significant transition in land-use activities. Large areas of natural forests have been converted into plantations, pastures and urban areas to provide resources for human beings. This has led to a significant loss in vegetation cover worldwide and serious environmental impacts; such as a loss of biodiversity and increase in greenhouse gas emissions. Similar losses in vegetation cover have affected the north Indian state of Himachal Pradesh, which is known to be one of the most ecologically rich states in the country.

\justify
The study area, Dharamshala Tehsil region, is located in the Indian state of Himachal Pradesh. Within the last decade, there has been an emergence of significant illegal logging and mining in the study area; as implied by various anecdotal sources. However, there is a lack of publicly accessible vegetation cover data to verify or quantify this evidence. Furthermore, there is also a lack of publicly accessible and spatially/temporally relevant auxiliary data from Indian-based and external organizations for us to investigate further. This study proposes an alternative vegetation monitoring strategy which enables effective spatio-temporal analysis of vegetation cover in the study area. The end goal of this study is to identify sub-regions in the study area from 2013-2017 which underwent significant vegetation loss. This information would be passed on to relevant authorities for further investigation.

\justify
In this study, remote sensing datasets were queried and Landsat 8 Surface Reflectance data was identified to be the most suitable for analysis. Image classification algorithms were queried and the random forests algorithm was identified as an effective supervised classifier. Field work was conducted to acquire field data of pre-defined vegetation classes; which would later be used for supervised classification. Supervised classification was then conducted to obtain 26 vegetation classification images of the study area. These images were then grouped and analyzed using a raster time-series framework and the Mann-Whitney $U$ test. With this, the study was able to identify vegetated sub-regions in the period of 2013-2017 which had undergone significant vegetation cover loss.

\justify
Despite certain limitations in the vegetation cover loss analysis, the proposed vegetation monitoring strategy in this study was deemed to be viable and effective in conducting spatio-temporal analyses of vegetation cover in the study area. 

\justify
\textbf{Keywords:} Remote Sensing, Vegetation Cover Monitoring, Landsat 8 Surface Reflectance Data, Random Forests Algorithm, Supervised Classification

\clearpage

\section*{Acknowledgements}
\addcontentsline{toc}{section}{Acknowledgements}

\justify
I am thankful to many people who offered support and generosity in helping make this research possible. First and foremost, my gratitude to Dr. Frank Molkenthin for supporting my research and for supervising this work despite other hurdles and stresses to manage. My gratitude to Dr. Wolfgang Preuss for supporting me with the intricacies of statistical analyses. 

\justify
My sincere thanks to my friend, Ashkan Mansouri Yarahmadi, for supporting me in formatting this work. I am grateful to the generosity of the vast online R, Google Earth Engine and LaTeX communities in StackExchange and other platforms; who never fail to offer their support to users facing problems. There have been countless issues fixed through your help.

\justify
In the area of modeling, my special thanks goes to Ali Santacruz, PhD candidate in the Geography program at Clark University (MA, USA), for providing critical assistance in customizing and programming the random forest model. Your patience and persistence is absolutely appreciated.

\justify
I am grateful to my mother and family in India, for providing unconditional support and care in finishing my research. This research would not have gone anywhere without your love and persistent support.

\justify
Lastly, I want to thank you, Miku, for your daily, hourly, temporally-infinitesimal support in completing this work. This research was inspired by you, your hard-work, your brilliance and of course, your \textit{gawa}. I am always grateful and driven by you. For all life brings, there is always greatness and brilliance around us. It is only a matter of time; time until the fruition we deserve.